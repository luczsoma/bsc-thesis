\chapter{Related Work}

This chapter specifies the currently known approaches and related work of static analysis in general, and specifically for JavaScript.


\section{Static analysis tools for JavaScript}

This section introduces several static analysis tools for the main subject of this thesis, the JavaScript language.


\subsection{TAJS (Type Analysis for JavaScript)}

TAJS is a static data flow analysis tool for JavaScript with the capability of inferring detailed and sound type information using abstract interpretation.~\cite{jensen2009type} In the time of this writing, it fully supports the 3\textsuperscript{rd} version of ECMAScript, and partially supports the 5\textsuperscript{th} version\footnote{ECMAScript 5 is the most popular, and most broadly used version of ECMAScript, supported by most of the desktop and mobile browsers and external runtimes.~\cite{kangax-es5} This is the ECMAScript version I referred to previously as \emph{plain JavaScript}.}, including its standard library, the HTML DOM, and the browser API.~\cite{tajs-github}

The abstract interpretation approach consists of the following main points:~\cite{tajs-presentation}

\begin{enumerate}
\item construct the \emph{Control-Flow Graph} of the program,
\item define a data flow lattice~\cite{jensen2009type}, which abstracts program data flow into a mathematically interpreted format,
\item define transfer functions, which abstracts the operations on the data flow lattice.
\end{enumerate}

There is an Eclipse plug-in for TAJS, but it is not ready for production usage, according to the creators of the framework.~\cite{tajs-website}


\subsection{Flow}

Flow is a static type checker for JavaScript developed and maintained by the Facebook Open Source community.~\cite{flow-github} Flow checks the code for defects based on \emph{static type annotations}.~\cite{flow-website} Without explicit type annotations, Flow is still able to work by attempting to infer types implicitly. Thus, into larger codebases, Flow can be introduced incrementally.

Like many other static analysis tools, Flow also aims for soundness, while preventing extensive reporting of false positives. The developers of the tool identified two main goals: precision and speed. According to the very imprecise documentation~\cite{flow-docs}, Flow is made to be practically precise by modeling the language's essential characteristics accurately enough to differentiate between intentional solutions and unintentional mistakes.

Flow's speediness means to be part of the editing process: the goal is to be fast enough for an IDE to show type information in real-time, during editing the code. To achieve this speed, Flow uses incremental processing with file-granularity, meaning only the changes since the last analysis need to be processed.

\subsection{Tern}

From the Tern website: \textquote{Tern is a stand-alone code-analysis engine for JavaScript. It is intended to be used with a code editor plugin to enhance the editor's support for intelligent JavaScript editing.}~\cite{tern-website}

Tern provides features like editor auto-completion of variables and properties, function argument hints, automatic refactoring, and finding the definition of a function or variable. It is written in JavaScript, and capable of running both on node.js and in the browser.

The software is maintained on GitHub~\cite{tern-github} by Marijn Haverbeke, developer of the Acorn lightweight JavaScript parser. Acorn is used as the underlying parser for the Tern infrastructure. The infrastructure itself consists of several components: the editor plugins communicate with the Tern server, which is implemented on top of the server module, which uses the inference engine to do the actual type inference.~\cite{tern-website}

Editor plug-ins' list contains editors with significant or growing popularity:

\begin{itemize}
\item Emacs
\item Vim
\item Sublime Text
\item Brackets
\item Eclipse
\end{itemize}

The newest version in the time of this writing is $0.21$, implying that the tool is not yet aimed for heavyweight production usage, but for experimental purposes.


\subsection{Shift}

Shift is not a static analysis tool, but a toolset created and developed by Shape Security, consisting of several tools.~\cite{shift-ast} Besides others, Shift features a parser, a code generator, and a scope analyser. It supports the full \emph{ECMAScript 7\textsuperscript{th} Edition}.~\cite{shift-ast}, and its parser and scope analyser is are foundations of \emph{Codemodel-Rifle}.

It is to be mentioned here, that Shift uses its own AST format, first announced by Shape Security before the Christmas of 2014, as their first open-source contribution. According to their reasoning, a new ECMAScript AST format was needed because its predecessor, Mozilla's SpiderMonkey AST was not specifically created for static analysis purposes, but rather for an internal representation only for interpretation.

Shift AST is said to comply with all aspects of a good AST-format~\cite{shift-ast-comparison}, as

\begin{itemize}
\item it minimizes the number of inhabitants that do not represent a program,
\item it is at least partially homogenous to allow for a simple and efficient visitor,
\item it does not impede moving, copying, or replacing subtrees,
\item it discourages duplication in code that operates on it.~\cite{shift-ast-comparison}
\end{itemize}

\subsection{Esprima}

Esprima is an ECMAScript parser with extended capabilities, like syntax validation. It supports the full standard of \emph{ECMAScript 7\textsuperscript{th} Edition}. The open-source software is created by Ariya Hidayat, engineer of \emph{Shape Security}, and is maintained on GitHub.


\subsection{Comparison of the featured tools}

\begin{table}[!h]
	\newcommand{\fullsupport}{\tikz\draw[black,fill=black] (0,0) circle (0.8ex);\xspace}
	\newcommand{\partialsupport}{\tikz\draw[black,fill=none] (0,0) circle (0.8ex);\xspace}
	\newcommand{\nosupport}{—}
	\centering
	\begin{tabular}{l|ccccc}
		\toprule
																					&   \textbf{TAJS}   &   \textbf{Flow}   &   \textbf{Tern}   &   \textbf{Shift Java}   \\
		\midrule
		\textbf{ECMAScript support}           &   ES3             &   ES5             &   ES7             &   ES7                   \\
		\textbf{AST-format}                   &   \nosupport      &   \nosupport      &   SpiderMonkey    &   Shift                 \\
		\textbf{open-source}                  &   \fullsupport    &   \fullsupport    &   \fullsupport    &   \fullsupport          \\
		\textbf{number of contributors}       &   1               &   335             &   87              &   10                    \\
		\textbf{licensing}                    &   Apache 2.0      &   BSD 3           &   MIT             &   Apache 2.0            \\
		\textbf{current version number}       &   v0.9-10         &   v0.45.0         &   0.21.0          &   es2016-v1.1.1         \\
		\midrule
		\textbf{infers types}                 &   \fullsupport    &   \fullsupport    &   \fullsupport    &   \nosupport            \\
		\textbf{needs non-standard syntax}    &   \nosupport      &   \fullsupport    &   \nosupport      &   \nosupport            \\
		\textbf{analyses related files}       &   \nosupport      &   \nosupport      &   \fullsupport    &   \nosupport            \\
		\bottomrule
	\end{tabular}

	\caption{Excerpt from an ECMAScript 6 compatibility table~\cite{kangax}}
	\label{table:javascript-tools-comparison}
\end{table}


\section{Static analysis tools for Java}

This section introduces static analysis tools for Java, mainly for earning new ideas regarding static analysis.


\subsection{FindBugs}

FindBugs is a static analysis tool for detecting bug patterns in Java code.~\cite{findbugs-website} One of its main techniques is to syntactically match source code to programming constructs marked as suspicious programming practise. \textquote{For example, FindBugs checks that calls to \texttt{wait()}, used in multi-threaded Java programs, are always within a loop–which is the correct usage in most cases. In some cases, FindBugs also uses dataflow analysis to check for bugs. For example, FindBugs uses a simple, intraprocedural (within one method) dataflow analysis to check for null pointer dereferences. FindBugs can be expanded by writing custom bug detectors in Java. We set FindBugs to report \textquote{medium} priority warnings, which is the recommended setting.}~\cite{rutar2004comparison}


\subsection{PMD}

Similar to FindBugs, PMD performs syntactic analysis on Java programs, but is does not have a data flow component. \textquote{In addition to some detection of clearly erroneous code, many of the \textquote{bugs} PMD looks for are stylistic conventions whose violation might be suspicious under some circumstances. For example, having a try statement with an empty catch block might indicate that the caught error is incorrectly discarded. Because PMD includes many detectors for bugs that depend on programming style, PMD includes support for selecting which detectors or groups of detectors should be run.}~\cite{rutar2004comparison}


\subsection{jQAssistant}

A German technology firm, Buschmais developed a component-based static analysis tool for Java, called jQAssistant.~\cite{jqassistant-website} Similarly to the Codemodel-Rifle framework, jQAssistant is built upon the Neo4j graph database. According to the documentation~\cite{jqassistant-documentation}, the tool is to be integrated into the build process to detect constraint violations and generate reports about user defined concepts and metrics.

Analysis rules can be expressed in Neo4j's graph query language, Cypher. However, instead of the semantics of the source code itself, jQAssistant focuses on the components and their connections. Its features include validating dependencies between modules in a project, enforcing naming conventions e.g. for test classes, packages, JPA entities, and detecting common architectural problems like cyclic dependencies.~\cite{jqassistant-documentation} The products is licensed under GNU General Public License, v3, allowing developers to use it in open-source projects.~\cite{gplv3}


\section{Static analysis tools for C and C++}


\subsection{Clang}

Besides serving as a compiler front-end for LLVM, Clang has a static analyser tool for finding bugs in C, C++, Objective-C programs.~\cite{clang-analyser-website} The tool can be used either as a standalone command-line tool, or as an Xcode\footnote{Xcode is Apple's integrated development environment only available for Apple's macOS, containing a suite of development tools for Apple platforms: macOS, iOS, watchOS and tvOS} plugin.

Clang uses static analysis based on compiler techniques. It is designed to report much more information than GCC, using control-flow graph analysis. It features flow- and path-sensitive analyses while preserving the overall form of the original source code.~\cite{kremenek2008finding} The tool can be integrated into IDEs, and supports automated refactoring.


\subsection{PolySpace}

PolySpace Technologies, which first developed the PolySpace Verifier, was later acquired by MathWorks. PolySpace Verifier has been reorganised into a suite, which now features static analysis for C and C++. Similar to TAJS's approach, PolySpace uses the classic lattice-theoretic abstract interpretation technique. The underlying analyser relies on a sound approximation of the set of all reachable states.~\cite{emanuelsson2008comparative} The tool features a data structure named \emph{convex polyhedron}, multiple \emph{convex polyhedra} encodes the sets of states.~\cite{cousot1978automatic} A convex polyhedron is an n-dimensional geometric shape where for any pair of points inside the shape the straight line connecting the points is also inside the shape.~\cite{emanuelsson2008comparative}

The tool is sound in the meaning that, given the full code base of the project, it computes the superset of every reachable state. It is flow-sensitive, context-sensitive, features inter-procedural analyses, and supports aliasing. \textquote{The properties checked by PolySpace Verifier are in many cases similar as those checked e.g. by other commercial systems, but the analysis is more sophisticated taking account of non-trivial relationships between variables (taking advantage of convex polyhedra) while other static analysis tools seem to cater only for simple relationships (e.g. equalities between variables and variables being bound to constant values or intervals of values).}~\cite{emanuelsson2008comparative}

PolySpace Verifier features checks for array conversion range extensions, return value initializations, variable initializations, pointer initializations, scalar/float under- and overflows and division by zero, non-termination of calls and loops, correctness of function arguments, unreachable code and many others.~\cite{emanuelsson2008comparative}

In today's product portfolio~\cite{polyspace-website}, Polyspace Bug Finder™ features the goal of locating defects with static analysis, and Polyspace Code Prover™ is said to prove the absence of run-time errors in C and C++ source code.

\subsection{Coverity}

Coverity Prevent, now part of Synopsys~\cite{coverity-website}, is a static analysis tool created as a spin-off from a research group at Stanford University. \textquote{In 2006 Coverity and Stanford were awarded a substantial grant from the U.S. Department of Homeland Security to improve Coverity tools to hunt for bugs and vulnerabilities in open-source software. During the first year 5,000 defects were fixed in some 50 open source projects. Updated results of the analyses can be found on the web.

The tool itself is a data-flow analysis tool featuring inter-procedural analyses. The analysis is neither sound nor complete, that is, there may be both defects which are not reported and there may be false alarms. A substantial effort has however been put on eliminating false positives, and the rate of these is clearly low (reportedly around 20 per cent).}~\cite{emanuelsson2008comparative}

Coverity features a different set of C and C++ checkers. For C, there are checks for resource leaks, dereferencing/deallocating already deallocated memory, uninitialised variables, unused pointer values, dead code, null pointer dereferences, misuse of negative integers and functions that may return negative integers, and null returns, amongst others. For C++, there are checks for errors in overriding virtual functions, resource leaks because of missing destructors, past-the-end STL iterators, and uncaught exceptions, amongst others.

Coverity has concurrency and security checkers as well, such as checks for double locks and missing releases, dangerous function calls like \texttt{gets} or \texttt{strcpy}, string overflows, and incorrect usage of the \texttt{chroot} system call.~\cite{emanuelsson2008comparative}