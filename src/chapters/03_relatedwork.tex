\chapter{Related Work}

This chapter specifies the currently known approaches and related work of static analysis in general, and specifically for JavaScript.


\section{Static analysis tools for JavaScript}

This section introduces several static analysis tools for the main subject of this thesis, the JavaScript language.


\subsection{TAJS (Type Analysis for JavaScript)}

TAJS is a static data flow analysis tool for JavaScript with the capability of inferring detailed and sound type information using abstract interpretation.~\cite{jensen2009type} In the time of this writing, it fully supports the 3\textsuperscript{rd} version of ECMAScript, and partially supports the 5\textsuperscript{th} version\footnote{ECMAScript 5 is the most popular, and most broadly used version of ECMAScript, supported by most of the desktop and mobile browsers and external runtimes.~\cite{kangax-es5} This is the ECMAScript version I referred to previously as \emph{plain JavaScript}.}, including its standard library, the HTML DOM, and the browser API.~\cite{tajs-github}

The abstract interpretation approach consists of the following main points:~\cite{tajs-presentation}

\begin{enumerate}
\item construct the \emph{Control-Flow Graph} of the program,
\item define a data flow lattice~\cite{jensen2009type}, which abstracts program data flow into a mathematically interpreted format,
\item define transfer functions, which abstracts the operations on the data flow lattice.
\end{enumerate}

There is an Eclipse plug-in for TAJS, but it is not ready for production usage, according to the creators of the framework.~\cite{tajs-website}


\subsection{Flow}

Flow is a static type checker for JavaScript developed and maintained by the Facebook Open Source community.~\cite{flow-github} Flow checks the code for defects based on \emph{static type annotations}.~\cite{flow-website} Without explicit type annotations, Flow is still able to work by attempting to infer types implicitly. Thus, into larger codebases, Flow can be introduced incrementally.

Like many other static analysis tools, Flow also aims for soundness, while preventing extensive reporting of false positives. The developers of the tool identified two main goals: precision and speed. According to the very imprecise documentation~\cite{flow-docs}, Flow is made to be practically precise by modeling the language's essential characteristics accurately enough to differentiate between intentional solutions and unintentional mistakes.

Flow's speediness means to be part of the editing process: the goal is to be fast enough for an IDE to show type information in real-time, during editing the code. To achieve this speed, Flow uses incremental processing with file-granularity, meaning only the changes since the last analysis need to be processed.

\subsection{Tern}

From the Tern website: \textquote{Tern is a stand-alone code-analysis engine for JavaScript. It is intended to be used with a code editor plugin to enhance the editor's support for intelligent JavaScript editing.}~\cite{tern-website}

Tern provides features like editor auto-completion of variables and properties, function argument hints, automatic refactoring, and finding the definition of a function or variable. It is written in JavaScript, and capable of running both on node.js and in the browser.

The software is maintained on GitHub~\cite{tern-github} by Marijn Haverbeke, developer of the Acorn lightweight JavaScript parser. Acorn is used as the underlying parser for the Tern infrastructure. The infrastructure itself consists of several components: the editor plugins communicate with the Tern server, which is implemented on top of the server module, which uses the inference engine to do the actual type inference.~\cite{tern-website}

Editor plug-ins' list contains editors with significant or growing popularity:

\begin{itemize}
\item Emacs
\item Vim
\item Sublime Text
\item Brackets
\item Eclipse
\end{itemize}

The newest version in the time of this writing is $0.21$, implying that the tool is not yet aimed for heavyweight production usage, but for experimental purposes.


\subsection{Shift}

Shift is not a static analysis tool, but a toolset created and developed by Shape Security, consisting of several tools.~\cite{shift-ast} Besides others, Shift features a parser, a code generator, and a scope analyser. It supports the full \emph{ECMAScript 7\textsuperscript{th} Edition}.~\cite{shift-ast}, and its parser and scope analyser is are foundations of \emph{Codemodel-Rifle}.

It is to be mentioned here, that Shift uses its own AST format, first announced by Shape Security before the Christmas of 2014, as their first open-source contribution. According to their reasoning, a new ECMAScript AST format was needed because its predecessor, Mozilla's SpiderMonkey AST was not specifically created for static analysis purposes, but rather for an internal representation only for interpretation.

Shift AST is said to comply with all aspects of a good AST-format~\cite{shift-ast-comparison}, as

\begin{itemize}
\item it minimizes the number of inhabitants that do not represent a program,
\item it is at least partially homogenous to allow for a simple and efficient visitor,
\item it does not impede moving, copying, or replacing subtrees,
\item it discourages duplication in code that operates on it.~\cite{shift-ast-comparison}
\end{itemize}

\subsection{Esprima}

Esprima is an ECMAScript parser with extended capabilities, like syntax validation. It supports the full standard of \emph{ECMAScript 7\textsuperscript{th} Edition}. The open-source software is created by Ariya Hidayat, engineer of \emph{Shape Security}, and is maintained on GitHub.


\subsection{Comparison of the featured tools}

\begin{table}[!h]
	\newcommand{\fullsupport}{\tikz\draw[black,fill=black] (0,0) circle (0.8ex);\xspace}
	\newcommand{\partialsupport}{\tikz\draw[black,fill=none] (0,0) circle (0.8ex);\xspace}
	\newcommand{\nosupport}{—}
	\centering
	\begin{tabular}{l|ccccc}
		\toprule
																					&   \textbf{TAJS}   &   \textbf{Flow}   &   \textbf{Tern}   &   \textbf{Shift Java}   \\
		\midrule
		\textbf{ECMAScript support}           &   ES3             &   ES5             &   ES7             &   ES7                   \\
		\textbf{AST-format}                   &   \nosupport      &   \nosupport      &   SpiderMonkey    &   Shift                 \\
		\textbf{open-source}                  &   \fullsupport    &   \fullsupport    &   \fullsupport    &   \fullsupport          \\
		\textbf{number of contributors}       &   1               &   335             &   87              &   10                    \\
		\textbf{licensing}                    &   Apache 2.0      &   BSD 3           &   MIT             &   Apache 2.0            \\
		\textbf{current version number}       &   v0.9-10         &   v0.45.0         &   0.21.0          &   es2016-v1.1.1         \\
		\midrule
		\textbf{infers types}                 &   \fullsupport    &   \fullsupport    &   \fullsupport    &   \nosupport            \\
		\textbf{needs non-standard syntax}    &   \nosupport      &   \fullsupport    &   \nosupport      &   \nosupport            \\
		\textbf{analyses related files}       &   \nosupport      &   \nosupport      &   \fullsupport    &   \nosupport            \\
		\bottomrule
	\end{tabular}

	\caption{Excerpt from an ECMAScript 6 compatibility table~\cite{kangax}}
	\label{table:javascript-tools-comparison}
\end{table}




\section{Static analysis tools for Java}


\subsection{FindBugs}


\subsection{jQAssistant}


\subsection{SonarQube}


\section{Static analysis tools for C and C++}


\subsection{Clang}


\subsection{PolySpace Verifier}


\subsection{Coverity}


\section{Most used general error-checking constraints}